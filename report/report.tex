\documentclass[]{article}

\usepackage{amsmath}  % AMS math package
\usepackage{amssymb}  % AMS symbol package
\usepackage{bm}       % bold math
\usepackage{graphicx} % Include figure files
\usepackage{dcolumn}  % Align table columns on decimal point
\usepackage{multirow} % Multirow/column tables
\usepackage{hyperref} % Hyperlinks

\begin{document}

\title{Modeling Ising}
\author{Scott O'Connor}
\date{\today} 
\maketitle

  \begin{abstract}
  The ising model is used in this program as a way to predict the behavior of magnetic material as it is effected by temperature. 
  \end{abstract}

\section{Program Explained}
  \subsection{}
  The first thing this model does is calculate the Hamiltonian.
  This is done by summing the states of each location in the Hamiltonian
  \begin{equation}
      H(\sigma) = - \sum_{i=1} \sigma_i \sigma_j 
  \end{equation}
  For a 2D case there are four possible nearest neighbor interations: the cell above, below, left and right.
  The lattice currently implimented is zeropadded. This allows for the ability to sum over the non-zero padded portion of the lattice and not have to worry about the boundries. 

  After the intial calacluaiton of the hamiltonian, a monte carlo simulation is performed on the lattice. In the monte carlo routine, radom bits are selected, and a matropolis test are formed on them.
  The metropolis test will determine if a bit should be fliped or not.

  \begin{equation}
      
  \end{equation}


\section{Magnetization at Different Temperatures} 
    \label{sec:level1} 

    \subsection{\label{sec:level2} Temperature at 2}

    \subsection{\label{sec:level2} Temperature at 3}

\section{Energy at Different Temperatures}

    \subsection{\label{sec:level2} Temperature at 2}

    \subsection{\label{sec:level2} Temperature at 3}

\section{Average Magnetization as a function of temperature}

    \url{http://en.wikipedia.org/wiki/Ising_model}
\end{document}
